\chapter{入门}
\label{chap:unix_for_beginners}
\marginpar{1}

什么是 UNIX? 从狭义上来说, 它是一个分时操作系统内核. 内核是一种计算机程序,
用于控制计算机资源, 并将其分配给用户. 有了内核, 用户就可以运行他们自己的程序;
内核控制着连接到主机上的外围设备 (磁盘, 终端, 打印机等); 内核还提供了一
个文件系统, 用于管理需要长期存放的数据, 比如程序, 文档.

从广义上来说, 除了内核, UNIX 还必须包含一系列必要的应用程序, 比如编译器, 编辑
器, 命令语言, 文件复制与打印程序等.

再说得宽泛些, UNIX 甚至还可以包括由用户自己开发的, 可以在系统中运行的程序,
比如文档展示工具, 统计分析工具, 和图形图像软件包等.

如何恰当地使用这些 UNIX 概念, 取决于读者所考虑的系统层面. 在本书中, 当我们提到
UNIX 时, 其上下文语境足以指明它的具体含义.

有时候, UNIX 系统可能比看起来更加难以使用 --- 初学者很难知道如何以最高的效率
使用 UNIX. 但幸运的是, 入门并不是很困难 --- 通过对几个简单的工具的了解, 读者
就可以马上开始使用 UNIX 系统. 这一章的目标是帮助读者尽可能快得开始使用 UNIX.
本章只是一个概览, 而非手册, 我们会在后面的章节再详细地介绍本章提到的大部分内
容. 本章介绍的内容有:

\begin{itemize}
    \item 基础 --- 登录与注销, 简单的命令, 纠正打字错误, 邮件, 终端间通信.
    \item 日常使用 --- 文件与文件系统, 打印文件, 目录, 常用命令.
    \item 命令解释器 (shell) --- 文件命简写, 输入与输出重定向, 管道, 设置
        擦除与终止字符, 定义命令搜索路径.
\end{itemize}
  
如果读者曾经使用过 UNIX 系统, 那么本章的大部分内容都会感到很熟悉, 此时读者可
以直接从第 \ref{chap:the_file_system} 章开始阅读.

\marginpar{2}
即使是在阅读这一章, 读者也需要准备一份 \textit{UNIX Programmer's Manual},
与其在书中重复手册的内容, 还不如让读者自己去阅读手册来得方便. 本书的目标并
非取代手册, 而是向读者展示如何有效地使用手册中所介绍的命令. 另外, 我们介绍
的内容可能与读者所用的系统稍有不同. 手册的开头有一份排序了的索引, 通过该索引
读者可以快速地找到自己想要的程序.

最后提个建议: 不要害怕实验. 对于初学者来说, 不太可能出现一些能够伤害到他人
或自己的意外, 所以请通过实践来学习书中的内容. 这一章很长, 最好的方式是每次
只阅读几页, 然后在自己的系统中尝试一遍.

\section{开始}
\label{sec:getting_started}

\subsection{终端与键盘输入的一些必要条件}
\label{subsec:some_prerequisites_about_terminals_and_typing}

为了避免解释计算机操作过程中的方方面面, 我们假设读者已经具备了计算机终端的
基础知识, 并且知道如何使用它们, 如何读者对下面的内容感到疑惑, 请咨询相应地
专家.

UNIX 工作在 \upeterm{全双工} (full duplex) 模式下, 也就是说用户通过键盘按下
的字符会被发送到系统中, 而系统也会把这些字符发送到终端上, 从而显示在屏幕中.
通常情况下, 这个 \upeterm{回显} (echo) 过程只是把字符直接复制到屏幕上, 但
是有时候, 比如用户在输入密码, 此时用户输入的字符就不会出现在屏幕上.

键盘上的大多数字符都是普通的可打印字符, 它们本身没有特殊的意义. 少数字符会
告诉计算机如何解释用户的输入, 其中最重要的是 RETURN 键. RETURN 键宣告一行
输入的结束, 然后系统通过把终端的光标移动到下一行的开始来回显刚输入的那行数据.
为了让系统解释用户输入的字符, 必须先按下 RETURN 键.

RETURN 键属于 \upeterm{控制字符} (control character). 控制字符是一种不可见的字
符, 它们可以控制终端输入与输出的某些行为. 在一台设置合理的终端上, RETURN
拥有一个属于自己的独立按键, 但是其他大多数控制字符就没有这种待遇, 作为替代,
在输入它们之前要先按下 CONTROL 键 (除了 CONTROL, 还可以叫作 CTL,
CNTL 或 CTRL), 然后再按下另一个按键, 通常是一个英文字母. 例如, RETURN 既可以通过
按 RETURN 键输入, 也可以通过先按下 CONTROL 键, 再按下 \texttt{m} 来输入. 因此,
RETURN
也可以称为 control-m, 写作 \upectl{m}. 其他的控制字符包括 \upectl{d} ---
通知程序输入已结束; \upectl{g} --- 通知终端响铃; \upectl{h}, 也叫退格键,
\marginpar{3}
可以用来纠正打字错误; \upectl{i}, 经常叫作制表符, 它把光标移动到下一个制表停
止位, 打字机用得比较多. 在 UNIX 系统中, 制表停止位之间相隔 8 个空格. 退格符
与制表符在终端上都有相应的按键.

另外两个具有特殊含义的按键是 DELETE (有时候也叫作 RUBOUT, 或其他一些简写形式)
与 BREAK (有时候也叫作 INTERRUPT). 在大多数 UNIX 系统中, DELETE 会马上停止
程序, 而不用等它运行结束, 有些系统通过 \upectl{c} 完成这个功能. 还有些系统,
根据终端的连接方式, BREAK 其实就是 DELETE 或 \upectl{c} 的同义词.

\subsection{一次 UNIX 会话}
\label{subsec:a_session_with_unix}

接下来, 我们将以一次用户同 UNIX 系统之间的对话作为开始, 为了方便, 对话中会加入
注解. 对于书中的全部例子, 用户输入的内容用等宽斜体表示, 系统的输出信息用等宽
字体, 注解用斜体.

\begin{upeshell}
#Establish a connection: dial a phone or turn on a switch as necessary.#
#Your system should say#
login: ^you^			#Type your name, then press RETURN#
Password:			#Your password won't be echoed as you type it#
You have mail.			#There's mail to be read after you log in#
$				#The system is now ready for your commands#
$				#Press RETURN a couple of times#
$ ^date^				#What is the date and time?#
Sun Sep 25 23:02:57 EDT 1983
$ ^who^				#Who's using the computer?#
jlb      tty0    Sep 25 13:59
you      tty2    Sep 25 23:01
mary     tty4    Sep 25 19:03
doug     tty5    Sep 25 19:22
egb      tty7    Sep 25 17:17
bob      tty8    Sep 25 20:48
$ ^mail^				#Read your mail#
From doug Sun Sep 25 20:53 EDT 1983
give me a call sometime monday

?				#RETURN moves on to the next message#
From mary Sun Sep 25 19:07 EDT 1983	#Next message#

? ^d^				#Delete this message#
$				#No more mail#
$ ^mail mary^			#Send mail to mary#
lunch at 12 is fine
^ctl-d^				#End of mail#
$				#Hang up phone or turn off terminal#
				#and that's the end#
\end{upeshell}
有时候, 这就是一次会话的全部内容, 当然, 用户可能还会做一些其他工作.
\marginpar{4}
本节的剩下部分会对上面的会话进行详细解释, 除此之外, 还将介绍几个能完成
实际工作的程序.

\subsection{登录}
\label{subsec:logging_in}

为了登录系统, 读者须向系统管理员申请用户名与密码. UNIX 支持多种终端, 但最
好是支持小写字母, 并区分大小写的终端. 如果读者的终端只支持大写字母 (比如
视频和可移植终端), 那么操作起来将会非常麻烦, 最好换一个终端.

在登录之前要确认设备的电路开关已经设置妥当: 区分大小写, 全双工, 以及管理员
建议的其他设置, 比如 \upeterm{波特率}. 尽你的一切所能为终端建立一个连接,
这中间可能还会涉及到拨号和翻转开关等操作. 无论如何, 系统最终应该输出:
\begin{upeshell}
login:
\end{upeshell}
如果输出了垃圾信息, 那可能是波特率设置得有误, 检查一下波特率和其他开关. 如果
还不行, 就缓慢地按几下 BREAK 和 INTERRUPT. 如果还是看不到登录信息, 就向管理
员寻求帮助.

看到 \texttt{login:} 后, 输入用户名, 再按下 RETURN. 如果还需要密码, 系统就
会要求用户输入密码, 在输入密码的过程中, 回显被关闭.

登录完成后将会看到 \upeterm{命令提示符} (prompt), 命令提示符通常就是一个字
符, 表示系统现在已经准备好接收用户的输入. 最常用作提示符的字符是美元符
\verb'$' 和百分号 \verb'%', 不过用户可以把提示符设置成任意内容, 稍后我们就
会介绍如何设置. 实际上, 打印提示符的是一个程序, 这个程序叫作
\upeterm{命令解释器} (command interpreter) 或 \textit{shell}, 它是用户同
系统交互的主要接口.

在提示符之前, 可能还会看到日期, 或者是通知用户有邮件未读. 系统可能还会询问
终端的类型, 用户的回答可以帮助系统设置终端的属性.

\subsection{键入命令}
\label{subsec:typing_commands}

看到提示符后就可以键入 \upeterm{命令} (command), 通过命令, 用户可以请求系统
做一些工作. 我们将使用 \upeterm{程序} (program) 作为命令的同义词. 看到提示
符 (假设提示符就是 \verb'$') 后, 键入 \texttt{date}, 再按下 RETURN, 系统就会
打印当前的日期和时间, 然后是另一个提示符. 在终端中, 整个事务的过程就像下面
这样:
\begin{upeshell}
$ ^date^
Mon Sep 26 12:20:57 EDT 1983
\end{upeshell}
不要忘记 RETURN, 也不要键入 \verb'$'. 如果用户觉得自己被系统忽略了,
\marginpar{5}
就按下 RETURN, 这时应该会有所变化. 在后面的内容中我们不再提及 RETURN,
但是用户要记得每输完一行后, 都要按下 RETURN.

下一个要介绍的命令是 \texttt{who}, 它会打印当前已登录的所有用户:
\begin{upeshell}
$ ^who^
rlm	tty0	Sep 26 11:17
pjw	tty4	Sep 26 11:30
gerard	tty7	Sep 26 10:27
mark	tty9	Sep 26 07:59
you	ttya	Sep 26 12:20
\end{upeshell}
第 1 列是用户名, 第 2 列是用户所使用的连接在系统中的名字 (tty 表示 teletype,
是 terminal 的同义词), 剩下的内容是登录时间. 读者可以试一下:
\begin{upeshell}
$ ^who am i^
you	ttya	Sep 26 12:20
$
\end{upeshell}

如果用户输错了命令的名字, 而错误的名字引用了一个不存在的命令, 那么系统就
会告知用户找不到对应的命令:
\begin{upeshell}
$ ^whom^				#Misspelled command name ...#
whom: not found			#... So system didn't know how to run it#
$
\end{upeshell}
当然, 如果用户不小心输入了另一个命令的名字, 那么命令也会正常执行, 只不过执行
结果可能跟想像中的不一样.

\subsection{奇怪的终端行为}
\label{subsec:strange_terminal_behavior}

有时候终端的表现可能会有异常, 比如每一个字符都显示两次, 又或者是 RETURN
没有把光标移动到下一行的第一列. 这些问题通常可以通过关闭再打开终端, 或注销
后再登录来解决, 另一种办法是阅读手册第 1 节中, 关于命令 \texttt{stty} (set
terminal options) 的介绍. 如果用户的终端没有制表符, 为了能让它恰当地处理
制表符, 键入:
\begin{upeshell}
$ ^stty -tabs^
\end{upeshell}
之后系统就会自动地把制表符转换成对应个数的空格. 如果用户的终端支持计算机可
配置的制表停止位, 命令 \texttt{tabs} 就可以帮你完成停止位的设置:
\begin{upeshell}
$ ^tabs^ #terminal-type#
\end{upeshell}
阅读手册以获取命令 \texttt{tabs} 的使用方法.

\subsection{纠正拼写错误}
\label{subsec:mistakes_in_typing}
\marginpar{6}

如果用户在按下 RETURN 之前发现自己的输入有误, 这时候有两种解决办法: 一个一个
地擦除 (erase) 字符, 或者是删除 (kill) 一整行, 然后重新输入.

如果用户输入了 \upeterm{删行} (line kill) 字符 (默认是 \verb'@'), 那么整行
都会被丢弃, 然后在新的一行开始:
\begin{upeshell}
$ ^ddtae@^	#Completely botched; start over on a new line#
^date^
Mon Sep 26 12:23:39 EDT 1983
$
\end{upeshell}

符号 \verb'#' 擦除最后一次输入的字符, 每一个 \verb'#' 都可以擦除一个对应的字
符. 如果用户输入的命令含有错误, 可以通过插入 \verb'#' 来纠正:
\begin{upeshell}
$ ^dd#atte##e^			#Fix it as you go#
Mon Sep 26 12:23:39 EDT 1983
$
\end{upeshell}

擦除和删行字符非常依赖于系统. 许多系统 (包括我们现在用的) 把擦除符改为退格
键, 它在视频终端中工作地非常好. 用户可以用下面的方法来判断系统用的是擦除符
还是退格键:
\begin{upeshell}
$ ^datee^+\textleftarrow+				#Try #+\textleftarrow+
$ datee+\textleftarrow+: not found		#It's not #+\textleftarrow+
$ ^datee#^				#Try #+\#+
Mon Sep 26 12:26:08 EDT 1983		#It is #+\#+
$
\end{upeshell}
(我们把退格键打印成 \textleftarrow, 这样读者就可以看到它们.) 另一种删除整行
的办法是 \upectl{u}.

因为 \verb'#' 是可打印字符, 所以在这节的剩下内容里, 我们将使用 \verb'#'
作为擦除字符, 如果读者的系统与我们不同, 就在键入时作相应的修改. 如果想输入
字面上的 \verb'#' 与 \verb'@', 键入 \verb'\#' 与 \verb'\@'. 输入 \verb'@'
后, 即使有前导的 \verb'\', 系统也有可能把终端的光标移动到下一行, 不用担心,
这时候系统已经读取到了 \verb'@'.

反斜杠, 有时候也称为 \upeterm{转义字符} (escape character), 被广泛地应用在
需要特殊对待下一个字符的场合中. 为了擦除一个反斜杠, 用户需要输入两个擦除字
符: \verb'\##'.

用户输入的字符在到达目的地之前, 会被一系列程序检查和解释. 解释的方式不仅取
\marginpar{7}
决于输入字符的目的地, 还和到达的方式有关.

用户键入的每一个字符都会立即回显在终端中, 除非终端关闭了回显, 不过这种情况
比较少见. 在按下 RETURN 之前, 键入的字符会暂时保留在内核里, 因此我们才能通过
擦除和删行字符来纠正拼写错误. 如果擦除或删行字符带有前导的反斜杠, 则内核会
丢弃反斜杠, 并不作解释地存放反斜杠的下一个字符.

按下 RETURN 后, 字符被发送给正在读取终端的程序, 这个程序可能会接着按照它自己
的方式对字符加以解释, 例如, 如果一个字符带有前导的反斜杠, 则 shell 就不会按
照特殊的方式解释该字符, 我们会在第 \ref{chap:using_the_shell} 章接着讨论这
部分内容. 现在读者只需要记住: 内核会处理擦除与删行字符, 以及带有前导反斜杠
的擦除与删行字符; 剩下的字符会接着被其他程序解释.

\begin{upeexer}
执行下面的命令, 推测出将要发生的现象:
\begin{upeshell}
$ ^date^+\textbackslash+^@^
\end{upeshell}
\end{upeexer}

\begin{upeexer}
大多数 shell (即使不是第 7 版的 shell) 把 \verb'#' 当成注释的开始, 它会忽略
从 \verb'#' 到行末的所有字符. 假设系统的擦除字符也是 \verb'#', 请根据前面
的话, 解释下面的命令执行结果:
\begin{upeshell}
$ ^date^
Mon Sep 26 12:39:56 EDT 1983
$ ^#date^
Mon Sep 26 12:40:21 EDT 1983
$ +\textbackslash+^#date^
$ +\textbackslash\textbackslash+^#date^
+\texttt{\#}+date: not found
$
\end{upeshell}
\end{upeexer}

\subsection{提前输入}
\label{subsec:type_ahead}

即使内核正在忙于处理其他事情, 它也会读取用户键入的字符, 所以说就算有其他命令
正在往终端上打印消息, 用户也可以照常输入. 如果在输入的同时, 系统正在打印消息,
那么用户的输入就会和终端的输出相混淆, 但是它们仍然可以准确无误地输送到内核中.
用户可以在上一条命令未结束, 甚至还未开始的情况下, 就可以接着输入下一条命令.

\subsection{停止程序}
\label{subsec:stopping_a_program}

键入 DELETE 字符可以停止大部分命令. 在大部分终端上, BREAK 键也可以完成相同的
功能, 但是具体的行为依赖于系统. 对于少数几个程序, 例如文本编辑器, DELETE
会停止程序, 但用户仍然会留在程序中, 不能离开. 关闭终端或挂断电话会停止大多数
\marginpar{8}
程序.

如果用户只是想暂停输出, 例如程序正在往终端打印一些重要的信息, 键入 \upectl{s},
输入就会马上停止, 程序也会被挂起, 直到用户输入 \upectl{q}, 程序才会继续执行.

\subsection{注销}
\label{subsec:logging_out}

注销登录比较恰当的做法是键入 \upectl{d}, 而不是执行某条命令. \upectl{d} 通知
shell 不会再有输入数据 (具体的过程在下一章介绍). 用户也可以关闭终端或挂断电
话, 但这是否会真正地注销登录则取决于系统.

\subsection{邮件}
\label{subsec:mail}

UNIX 提供了用于和其他用户通信的邮件系统, 所以用户在登录后, 可能会看到这样一
条消息:
\begin{upeshell}
You have a mail.
\end{upeshell}
为了读取邮件, 键入
\begin{upeshell}
$ ^mail^
\end{upeshell}
邮件会陆续打印出来, 一次打印一封, 后收到的邮件先打印. 打印完一封邮件后,
\texttt{mail} 等待用户的响应, 比较常用的两个响应命令是 \texttt{d} --- 删除
邮件; 以及 RETURN --- 不删除. 其他的响应命令还有 \texttt{p} --- 重新打印邮件;
\texttt{s }\textit{filename} --- 把邮件保存到以 \texttt{filename} 命名的文件
中 (如果读者不清楚什么是文件, 可以把它当作一个可以存放数据的地方, 它有一
个名字, 通过这个名字就可以找到对应的文件. 文件在
\ref{subsec:day_to_day_use_files_and_common_commands} 节介绍, 它是本书的重
点概念之一).

\texttt{mail} 命令有多种变体, 用户所用的 \texttt{mail} 可能和我们这里描述的
不一样, 具体的使用方法请参考相应的手册.

向某人发送邮件的过程非常简单直接, 假设我们要向用户 \texttt{nico} 发送一封邮
件, 最简单的办法是:
\begin{upeshell}
$ ^mail nico^
#Now type in the text of the letter#
#on as many line as you like ...#
#After the last line of the letter#
#type a control-d#
+\textit{\textrm{ctl}}+^-d^
$
\end{upeshell}
\upectl{d} 通知 \texttt{mail} 输入已经结束. 如果用户改变了心意, 键入 DELETE,
而非 \upectl{d}, 这时未写完的邮件内容会存放在名为 \texttt{dead.letter} 的
文件中, 不会发给 \texttt{nico}.

\marginpar{9}
用户可以试着给自己发一封邮件, 然后再用 \texttt{mail} 读取 (虽然这个做法听起
来有点奇怪, 但是能起到备忘录的作用).

发送邮件还有很多种办法 --- 用户可以发送一封已经准备好了的信, 可以一次发给多
个人, 还可以发给其他主机上的用户. \texttt{mail} 命令的所有细节都可以在
\texttt{UNIX Programmer's Manual} 的第 1 节找到, 今后, 我们将用 \texttt{mail}
(1) 表示手册的第 1 节中, 描述 \texttt{mail} 的内容, 本章介绍的所有命令都在
手册的第 1 节.

UNIX 系统还提供日历服务 (见 \texttt{calendar}(1)), 我们将在第
\ref{chap:filters} 章介绍如何自己开发一个日历工具.

\subsection{向其他用户发送信息}
\label{subsec:writing_to_other_users}

如果系统中同时有多个用户登录, 有一天, 你的终端可能会突然冒出类似下面的消息
\begin{upeshell}
Message from mary tty7...
\end{upeshell}
同时伴有一声响铃. 这表示用户 \texttt{mary} 想给你发送消息, 除非采取了显式
的动作, 否则你无法回应 \texttt{mary}. 为了回应消息, 键入
\begin{upeshell}
$ ^write mary^
\end{upeshell}
命令建立了一条双向的通信路径. 现在, Mary 在终端上输入的字符会同时显示在你的
终端上, 反之亦然, 但是通信速度会比较慢.

如果用户正处于程序运行的过程中, 但是此时你偏偏想要执行一条 shell 命令, 通
常来说这时候就必须先让程序停止运行, 但是有一些程序 --- 例如文本编程器和
\texttt{write} --- 它们提供了 `!' 命令, 用于临时跳脱到 shell 环境, 见附录
\ref{apd:editor_summary} 的表 \ref{tabl:summary_of_ed_commands}.

\texttt{write} 不对通信施加任何限制, 所以为了避免消息互相扰乱, 有必要遵守
一些协议. 一个常用的规则是一条消息输入完毕后, 在末尾添加 \texttt{(o)}, 表示
``over''; 如果想要退出会话, 就发送 \texttt{(oo)}, 表示 ``over and out''.
\marginpar{10}
\begin{upeshell}
#Mary's terminal:#			#Your terminal#
$ ^write you^
					$ Message from mary tty7...
					^write mary^
Message from you ttya...
^did you forget lunch? (o)^
					did you forget lunch? (o)
					^five@^
					^ten minutes (o)^
ten minutes (o)
^ok (oo)^
					ok (oo)
					+\textit{\textrm{ctl}}+^-d^
EOF
+\textit{\textrm{ctl}}+^-d^
					$ EOF
$
\end{upeshell}
退出 \texttt{write} 也可以通过按 DELETE 键完成. 注意, 用户的打字错误并不会出
现在另一个用户的终端上.

如果目标用户还没有登录系统, 或者是他不想被打扰, 那么发送消息的用户就会看到相
应的通知. 如果目标用户已经登录, 但是过了一会儿仍然没有回应消息, 他可能一时不
在, 也可能是正在忙于其他事情, 此时用户只需要按下 \upectl{d} 或 DELETE 键.
如果用户不想被打扰, 就用 \texttt{mesg}(1).

\subsection{新闻}
\label{subsec:news}

为了帮助用户阅读感兴趣的文章, 许多 UNIX 系统都提供了新闻服务, 使用新闻服务的
命令是
\begin{upeshell}
$ ^news^
\end{upeshell}
许多 UNIX 系统通过电话联结成一个巨大的网络, 向附近的专家咨询关于
\texttt{netnews} 与 USENET 的事情.

\subsection{手册}
\label{subsec:the_manual}

用户需要了解的关于 UNIX 系统的绝大部分知识, 都可以在 \textit{UNIX Programmer's
Manual} 中找到. 手册的第 1 节描述命令, 包括本章介绍的命令; 第 2 节描述系统
调用, 它是本书第 \ref{chap:unix_system_calls} 章的主要内容; 第 6 节描述游戏.
手册的其他部分描述 C 库函数, 文件格式, 以及系统管理 (在不同的系统手册中, 章节
的编号可能有所不同). 不要忘记在手册的开头还有一个索引, 索引可以帮助用户快速找
到感兴趣的命令. 手册还包含了一个关于系统如何工作的简短介绍.

通常情况下该手册是在线存放的, 因此用户可以直接在终端中阅读. 如果用户遇到问题,
又找有到专家咨询, 这时候应该执行 \texttt{man} \textit{command-name}, 打开命令
的手册页.
\marginpar{11}
例如, 为了阅读命令 \texttt{who} 的手册, 键入
\begin{upeshell}
$ ^man who^
\end{upeshell}
同样, 阅读 \texttt{man} 手册的方式是
\begin{upeshell}
$ ^man man^
\end{upeshell}

\subsection{计算机辅助教程}
\label{subsec:computer_aided_instruction}

用户的系统可能有命令 \texttt{learn}, 它提供了一些计算机辅助教程, 包括文件
系统, 基本命令, 编辑器, 文件准备, 甚至 C 程序设计, 打开教程的命令是
\begin{upeshell}
$ ^learn^
\end{upeshell}
如果系统有该命令, 它就会指导用户如何开始, 如果没有 \texttt{learn}, 尝试
\texttt{teach}.

\section{日常使用: 文件和常用命令}
\label{sec:day_to_day_use_files_and_common_commands}

UNIX 系统的信息都存放在 \upeterm{文件} (file) 中, 它就像办公室里的普通文件.
每个文件都有它自己的名字, 内容, 存放位置, 还有一些管理信息, 例如该文件属于
谁, 有多大. 文件的内容可以是一个字母, 一张地址表, 程序的源代码, 某个程序的输
入数据, 可执行程序, 以及其他的非文本格式数据.

UNIX 的文件系统保证了用户在管理自己的私有文件时, 不会干扰属于其他用户的文件,
同样, 其他用户也不会干扰你的私有文件. 管理文件的命令非常多, 但现在我们只介
绍最常用的命令. 第 \ref{chap:the_file_system} 章将对文件系统进行系统性的介绍,
到那时我们就会介绍更多的关于文件的命令.

\subsection{创建文件 --- 编辑器}
\label{subsec:creating_files_the_editor}

如果用户想要写一篇文档, 或开发一个程序, 那么如何才能把信息存放到机器中? 这
些工作通常可以用 \upeterm{文本编辑器} (text editor) 完成, 它可以存放并操作计
算机中的信息. 大部分 UNIX 系统都至少有一个 \upeterm{屏幕编辑器} (screen
editor), 这类编辑器利用现代终端的能力, 在屏幕上即时显示用户的每一次修改所产
生的效果, 其中最流行的两款编辑器是 \texttt{vi} 和 \texttt{emacs}. 我们不在这
\marginpar{12}
里介绍任意一种特定的屏幕编辑器, 一方面是由于印刷条件限制, 另一方面是因为不存
在一种标准的屏幕编辑器.

不过, 任意一个 UNIX 系统必定提供了编辑器 \texttt{ed}. \texttt{ed} 不依赖特定
的终端功能, 所以它可以在任意类型的终端中工作. \texttt{ed} 还是某些关键程序的
基础软件 (包括一些屏幕编辑器), 所以学习如何使用 \texttt{ed} 还是很有价值的.
附录 \ref{chap:editor_summary} 包含了 \texttt{ed} 的详细介绍.

不管用户用哪一种编辑器, 都需要学习如何用它创建文件. 为了具体一点, 我们以
\texttt{ed} 为例进行说明, 从而确保用户可以在自己的系统重复我们的操作. 但是,
如果用户有自己喜欢的编辑器, 应该尽一切办法学习如何使用它.

为了用 \texttt{ed} 创建一个文件 \texttt{junk}, 并输入一些文本, 执行:
\begin{upeshell}
$ ^ed^				#Invokes the text editor#
^a^				+\texttt{ed}+# command to add text#
#now type in
whatever text you want ...#
^.^				# Type a `.' by itself to stop adding text#
^w junk^			# Write your text into a file called junk#
39				+\texttt{ed}+# prints number of characters written#
^q^				#Quit #+\texttt{ed}+
$
\end{upeshell}
命令 \texttt{a} (``append'') 告诉 \texttt{ed} 开始接收文本. ``.'' 标志文本
结束, 它必须单独占据一行, 在键入单独的句点之间, 所有的 \texttt{ed} 命令都不
会被识别, 它们都会被当成普通的文本.

命令 \texttt{w} 把用户输入的文本存放到文件中, ``\texttt{w junk}'' 把前面输入
的内容存放到名为 \texttt{junk} 的文件中. 文件名可以按照用户的意愿随意命名,
我们把它叫到 \texttt{junk} 是为了说明该文件并不重要.

写完文件后, \texttt{ed} 打印写到文件中的字符个数. 在执行 \texttt{w} 之前, 前
面输入的文本不会写到文件中, 所以如果用户中途挂起会话, 然后回家去了, 那么文本
就不会写到文件中. (如果用户正在编辑时连接挂断, 那么前面输入的内容会存放到名
为 \texttt{ed.hup} 的文件中, 这样用户就可以在下一次会话时继续之前的工作)
如果在编辑时系统崩溃 (由于软件或硬件故障, 系统停止工作), 那么文件只含有崩溃
前最后一次执行 \texttt{w} 命令时写进去的内容. 一旦执行 \texttt{w}, 文本就会
永久地记录在文件中, 以后用户可以用下面的命令访问文件中的内容:
\begin{upeshell}
$ ^ed junk^
\end{upeshell}

当然, 用户也可以修改之前输入的文本. 等一切都做完后, 命令 \texttt{q} (``quit'')
退出编辑器.

\marginpar{13}
\subsection{目录中有哪些文件?}
\label{subsec:what_files_are_out_there}

创建 2 个文件, 分别叫做 \texttt{junk} 和 \texttt{temp}:
\begin{upeshell}
$ ^ed^
^a^
^To be or not to be^
^.^
^w junk^
19
^q^
$ ^ed^
^a^
^That is the question.^
^.^
^w temp^
22
^q^
$
\end{upeshell}
\texttt{ed} 计算的字符数包括每行末尾的 \upeterm{换行符} (newline), 系统用它
来表示回车键.

命令 \texttt{ls} 列出目录中的文件 (不包括文件中的内容):
\begin{upeshell}
$ ^ls^
junk
temp
$
\end{upeshell}
上面列出的就是我们刚刚创建的 2 个文件 (可能还有其他用户创建的文件). 文件名
自动按照字母顺序排列.

和大多数命令一样, \texttt{ls} 提供了很多用于改变默认行为的 \upeterm{选项}
(option). 在命令行上, 选项跟在命令的后面. 选项通常由一个负号 `-' 和一个表示
选项意义的字母组成. 例如, \texttt{ls -t} 使得文件名按照文件的最后修改时间
进行排序, 最近修改的文件先出现.
\begin{upeshell}
$ ^ls -t^
temp
junk
$
\end{upeshell}
选项 \texttt{-l} 使 \texttt{ls} 列出更详细的信息:
\begin{upeshell}
$ ^ls -l^
total 2
-rw-r--r-- 1 you	19 Sep 26 16:25 junk
-rw-r--r-- 1 you	22 Sep 26 16:26 temp
$
\end{upeshell}

\marginpar{14}
``\texttt{total 2}'' 指出了文件所消耗的磁盘块个数, 一个磁盘块通常是 512 或
1024 个字节. 字串 \texttt{-rw-r--r--} 指出了各类用户对文件的访问权限, 在这
个例子中, 文件所有者 (\texttt{you}) 可以读写文件, 但是其他用户只能读文件. 随
后的 ``\texttt{1}'' 指出了文件的链接数, 我们会在第 \ref{chap:the_file_system}
章介绍链接. ``\texttt{you}'' 是文件的所有人, 也就是创建该文件的用户.
\texttt{19} 和 \texttt{22} 是文件中包含的字符个数, 它们和 \texttt{ed} 保存文
件时打印的字符个数相同. 日期和时间指出了文件最后一次修改是在什么时候.

选项可以组合: \texttt{ls -lt} 打印的内容和 \texttt{ls -l} 相同, 但是文件名
按照最后修改时间排序. 选项 \texttt{-u} 指出了文件最后一次被访问是在什么时候:
\texttt{ls -lut} 输出一个长 (\texttt{-l}) 列表, 最近被访问的文件先打印.
用户也可以在命令的后面跟上文件名, 此时 \texttt{ls} 只会打印该文件的信息:
\begin{upeshell}
$ ^ls -l junk^
-rw-r--r-- 1 you	19 Sep 26 16:25 junk
$
\end{upeshell}

跟在程序名后面的字串, 例如上面例子中的 \texttt{-l} 和 \texttt{junk}, 叫做
程序的 \upeterm{参数} (argument). 参数通常是命令要用到的选项或文件名.

用负号和一个字母来表示选项是一种很常见的做法, 例如 \texttt{-t} 和组合后的
\texttt{-lt}. 一般情况下, 如果命令支持这种参数, 那么这些参数应该放在所有文
件名参数的前面, 但是选项之间可以以任意的次序出现. 但是不同的程序对组合选项
的处理可能有所不同. 例如, 第 7 版不认为
\begin{upeshell}
$ ^ls -l -t^		#Doesn't work in 7th Edition#
\end{upeshell}
是 \texttt{ls -lt} 的同义词, 还有些程序要求多个选项之间必须分开.

随着阅读的深入, 读者将会看到对于可选参数来说没什么规律可言. 每一个命令都有自
己的参数习惯, 字母选择也很自由 (即使是不同命令的同一功能, 表示参数的字母也不
尽相同). 这种不规律性经常使人困惑, 也常常被人当作 UNIX 系统的缺点. 虽然情况
正在改善 (新版本系统通常具有更强的一致性), 但我们建议读者在写程序时, 手上最
好要有一本手册.

\subsection{打印文件 --- \texttt{cat} 和 \texttt{pr}}
\label{subsec:printing_files_cat_and_pr}

现在读者手上已经有了几个文件, 那么应该如何查看文件当中的内容? 有很多程序可以
做到, 也许将来还会更多. 其中一种办法是使用编辑器:
