\chapter{入门}
\label{chap:unix_for_beginners}

什么是 UNIX? 从狭义上来说, 它是一个分时操作系统内核. 内核是一种计算机程序,
用于控制计算机资源, 并将其分配给用户. 有了内核, 用户就可以运行他们自己的程序;
内核控制着连接到主机上的外围设备 (磁盘, 终端, 打印机等); 内核还提供了一
个文件系统, 用于管理需要长期存放的数据, 比如程序, 文档.

从广义上来说, 除了内核, UNIX 还必须包含一系列必要的应用程序, 比如编译器, 编辑
器, 命令语言, 文件复制与打印程序等.

再说得宽泛些, UNIX 甚至还可以包括由用户自己开发的, 可以在系统中运行的程序,
比如文档展示工具, 统计分析工具, 和图形图像软件包等.

如何恰当地使用这些 UNIX 概念, 取决于读者所考虑的系统层面. 在本书中, 当我们提到
UNIX 时, 其上下文语境足以指明它的具体含义.

有时候, UNIX 系统可能比看起来更加难以使用 --- 初学者很难知道如何以最高的效率
使用 UNIX. 但幸运的是, 入门并不是很困难 --- 通过对几个简单的工具的了解, 读者
就可以马上开始使用 UNIX 系统. 这一章的目标是帮助读者尽可能快得开始使用 UNIX.
本章只是一个概览, 而非手册, 我们会在后面的章节再详细地介绍本章提到的大部分内
容. 本章介绍的内容有:

\begin{itemize}
    \item 基础 --- 登录与注销, 简单的命令, 纠正打字错误, 邮件, 终端间通信.
    \item 日常使用 --- 文件与文件系统, 打印文件, 目录, 常用命令.
    \item 命令解释器 (shell) --- 文件命简写, 输入与输出重定向, 管道, 设置
        擦除与终止字符, 定义命令搜索路径.
\end{itemize}
  
如果读者曾经使用过 UNIX 系统, 那么本章的大部分内容都会感到很熟悉, 此时读者可
以直接从第 \ref{chap:the_file_system} 章开始阅读.

即使是在阅读这一章, 读者也需要准备一份 \textit{UNIX Programmer's Manual},
与其在书中重复手册的内容, 还不如让读者自己去阅读手册来得方便. 本书的目标并
非取代手册, 而是向读者展示如何有效地使用手册中所介绍的命令. 另外, 我们介绍
的内容可能与读者所用的系统稍有不同. 手册的开头有一份排序了的索引, 通过该索引
读者可以快速地找到自己想要的程序.

最后提个建议: 不要害怕实验. 对于初学者来说, 不太可能出现一些能够伤害到他人
或自己的意外, 所以请通过实践来学习书中的内容. 这一章很长, 最好的方式是每次
只阅读几页, 然后在自己的系统中尝试一遍.

\section{开始}
\label{sec:getting_started}

\subsection{终端与键盘输入的一些必要条件}
\label{subsec:some_prerequisites_about_terminals_and_typing}

为了避免解释计算机操作过程中的方方面面, 我们假设读者已经具备了计算机终端的
基础知识, 并且知道如何使用它们, 如何读者对下面的内容感到疑惑, 请咨询相应地
专家.

UNIX 工作在 \upeterm{全双工} (full duplex) 模式下, 也就是说用户通过键盘按下
的字符会被发送到系统中, 而系统也会把这些字符发送到终端上, 从而显示在屏幕中.
通常情况下, 这个 \upeterm{回显} (echo) 过程只是把字符直接复制到屏幕上, 但
是有时候, 比如用户在输入密码, 此时用户输入的字符就不会出现在屏幕上.

键盘上的大多数字符都是普通的可打印字符, 它们本身没有特殊的意义. 少数字符会
告诉计算机如何解释用户的输入, 其中最重要的是 RETURN 键. RETURN 键宣告一行
输入的结束, 然后系统通过把终端的光标移动到下一行的开始来回显刚输入的那行数据.
为了让系统解释用户输入的字符, 必须先按下 RETURN 键.

RETURN 键属于 \upeterm{控制字符} (control character). 控制字符是一种不可见的字
符, 它们可以控制终端输入与输出的某些行为. 在一台设置合理的终端上, RETURN
拥有一个属于自己的独立按键, 但是其他大多数控制字符就没有这种待遇, 作为替代,
在输入它们之前要先按下 CONTROL 键 (除了 CONTROL, 还可以叫作 CTL,
CNTL 或 CTRL), 然后再按下另一个按键, 通常是一个英文字母. 例如, RETURN 既可以通过
按 RETURN 键输入, 也可以通过先按下 CONTROL 键, 再按下 \texttt{m} 来输入. 因此,
RETURN
也可以称为 control-m, 写作 \upectl{m}. 其他的控制字符包括 \upectl{d} ---
通知程序输入已结束; \upectl{g} --- 通知终端响铃; \upectl{h}, 也叫退格键,
可以用来纠正打字错误; \upectl{i}, 经常叫作制表符, 它把光标移动到下一个制表停
止位, 打字机用得比较多. 在 UNIX 系统中, 制表停止位之间相隔 8 个空格. 退格符
与制表符在终端上都有相应的按键.

另外两个具有特殊含义的按键是 DELETE (有时候也叫作 RUBOUT, 或其他一些简写形式)
与 BREAK (有时候也叫作 INTERRUPT). 在大多数 UNIX 系统中, DELETE 会马上停止
程序, 而不用等它运行结束, 有些系统通过 \upectl{c} 完成这个功能. 还有些系统,
根据终端的连接方式, BREAK 其实就是 DELETE 或 \upectl{c} 的同义词.

\subsection{一次 UNIX 会话}
\label{subsec:a_session_with_unix}

接下来, 我们将以一次用户同 UNIX 系统之间的对话作为开始, 为了方便, 对话中会加入
注解. 对于书中的全部例子, 用户输入的内容用等宽斜体表示, 系统的输出信息用等宽
字体, 注解用斜体.

\begin{verbatim}
        Establish a connection: dial a phone or turn on a switch as necessary.
        Your system should say
        login: you                 Type your name, then press RETURN
        Password:                  Your password won't be echoed as you type it
        You have mail.             There's mail to be read after you log in
        $                          The system is now ready for your commands
        $                          Press RETURN a couple of times
        $ date                     What is the date and time?
        Sun Sep 25 23:02:57 EDT 1983
        $ who                      Who's using the computer?
        jlb      ttyO    Sep 25 13:59
        you      tty2    Sep 25 23:01
        mary     tty4    Sep 25 19:03
        doug     tty5    Sep 25 19:22
        egb      tty7    Sep 25 17:17
        bob      tty8    Sep 25 20:48
        $ mail                     Read your mail
        From doug Sun Sep 25 20:53 EDT 1983
        give me a call sometime monday

        ?                          RETURN moves on to the next message
        From mary Sun Sep 25 19:07 EDT 1983     Next message

        ? d                        Delete this message
        $                          No more mail
        $ mail mary                Send mail to mary
        lunch at 12 is fine
        ctl-d                      End of mail
        $                          Hang up phone or turn off terminal
                                   and that's the end
\end{verbatim}
Sometimes that's all there is to a session, though occasionally people do some
work too. The rest of this section will discuss the session above, plus other
programs that make it possible to do useful things.


\subsection{Logging in}

You must have a login name and password, which you can get from your system
administrator. The UNIX system is capable of dealing wit a wide variety of
terminals, but it is strongly oriented towards devices with lower case; case
distinctions matter! If your terminal produces only upper case (like some video
and portable terminals), life will be so difficult that you should look for
another terminal.

Be sure the switches are set appropriately on your device: upper and lower case,
full duplex, and any other settings that local experts advise, such as the
speed, or \textit{baud rate}. Establish a connection using whatever magic is
needed for your terminal; this may involve dialing a telephone or merely
flipping a switch. In either case, the system should type
\begin{verbatim}
        login:
\end{verbatim}
If it types garbage, you may be at the wrong speed; check the speed setting and
other switches. If that fails, press the BREAK or INTERRUPT key a few times,
slowly. If nothing produces a login message, you will have to get help.

When you get the \verb=login:= message, type your login name \textit{in lower
  case}. Follow it by pressing RETURN. If a password is required, you will be
asked for it, and printing will be turned off while you type it.

The culmination of your login efforts is a \textit{prompt}, usually a single
character, indicating that the system is ready to accept commands from you. The
prompt is mostly likely to be a dollar sign \verb=$= or a percent sign \verb=%=,
but you can change it to anything you like; we'll show you how a little
later. The prompt is actually printed by a program called the \textit{command
  interpreter} or \textit{shell}, which is your main interface to the system.

There may be a message of the day just before the prompt, or a notification that
you have mail. You may also be asked what kind of terminal you are using; your
answer helps the system to use any special properties the terminal might have.


\subsection{Typing commands}

Once you receive the prompt, you can type commands, which are requests that the
system do something. We will use \textit{program} as a synonym for
\textit{command}. When you see the prompt (let's assume it's \verb=$=), type
date and press RETURN. the system should replay with the date and time, then
print another prompt, so the whole transaction will look like this on your
terminal:
\begin{verbatim}
        $ date
        Mon Sep 26 12:20:57 EDT 1983
        $
\end{verbatim}
Don't forget RETURN, and don't type the \verb=$=. If you think you're being
ignored, press RETURN; something should happen. RETURN won't be mentioned again,
but you need it at the end of every line.

The next command to try is who, which tells you everyone who is currently logged
in:
\begin{verbatim}
        $ who
        rim     tty0     Sep 26 11:37
        pjw     tty4     Sep 26 11:30
        gerard  tty7     Sep 26 10:27
        mark    tty9     Sep 26 07:59
        you     ttya     Sep 26 12:20
        $
\end{verbatim}
The first column is the user name. The second is the system's name for the
connection being used (``tty'' stands for ``teletype,'' an archaic synonym for
``terminal''). The rest tells when the user logged on. You might also try
\begin{verbatim}
        $ who am i
        you     ttya     Sep 26 12:20
        $
\end{verbatim}

If you make a mistake typing the name of a command, and refer to a non-existent
command, you will be told that no command of that name can be found:
\begin{verbatim}
        $ whom                     Misspelled command name ...    
        whom: not found            ... so system didn't know how to run it
        $
\end{verbatim}
Of course, if you inadvertently type the name of an actual command, it will run,
perhaps with mysterious results.


\subsection{Strange terminal behavior}

Sometimes your terminal will act strangely, for example, each letter may be
typed twice, or RETURN may not put the cursor at the first column of the next
line. You can usually fix this by turning the terminal off and on, or by logging
out and logging back in. Or you can read the description of the command
\texttt{stty} (``set terminal options'') in Section 1 of the manual. To get
intelligent treatment of tab characters if your terminal doesn't have tabs, type
the command
\begin{verbatim}
        $ stty -tabs
\end{verbatim}
and the system will convert tabs into the right number of spaces. If your
terminal does have computer-settable tab stops, the command tabs will set them
correctly for you. (You may actually have to say
\begin{verbatim}
        $ tabs terminal-type
\end{verbatim}
to make it work --- see the \texttt{tabs} command description in the manual.)


\subsection{Mistakes in typing}
\subsection{Type-ahead}
\subsection{Stopping a program}
\subsection{Logging out}
\subsection{Mail}
\subsection{Writing to other users}
\subsection{News}
\subsection{The manual}
\subsection{Computer-aided instruction}
\subsection{Games}

\section{Day-to-day use: files and common commands}
\section{More about files: directories}
\section{The shell}
\section{The rest of the UNIX system}
